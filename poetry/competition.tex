%\documentclass[fontsize=12pt,english]{article}
\documentclass[fontsize=12pt,english]{scrreprt}

\usepackage[utf8]{inputenc}
\usepackage[T1]{fontenc}
\usepackage[
  protrusion=true,
  expansion=true,
  tracking=true,
  kerning=true,
  spacing=true]{microtype}

% Get rid of the section numbers
\setcounter{secnumdepth}{0}

\title{Harbouring Hate}
\author{}
\date{\today}

\begin{document}

\maketitle
\tableofcontents

% What follows is a work of fiction. All characters and events
% portrayed herein are either fictitious or used fictitiously. Any
% similarity to any real people or places is purposeful, but not
% intended to be accurate, except where it is

\newpage

%306 716 5198
% Dr gavelis

\section{The Psychologist}

Looking over at Steve -- who was having his usual chatty morning,
standing by the window, soaking in the sun -- I wondered if he was
having a good conversation. Was it interesting? He seemed to laugh a
lot, so it must be good. Right? Looking back, I think I must have
envied his ability to talk to someone at any time. What I didn't envy
was that which allowed him to do so. I considered wandering over and
saying hi, but the politness in me always stoped me. I'm not sure if I
would ever have had the right timing -- maybe no one would.

The problem with schizophrenics is you never know if you're
interrupting.

\newpage

\section{Suddenly Saskatchewan}

Suddenly Saskatchewan -- the land of the living skies.

The shock of the prairies was visceral and deep. He was half way from
home: half way from his new home and half way from his old one. He
wasn't far enough to miss his old one, but he was close enough to be
nervous of his new one.

He'd opted to leave everything behind. His things were packed into a
small cube in the fringes of an old industrial park. His friends in
their homes. His former life in a bag in the first available garbage.

He carried on, not thinking to deeply about what would come along.

He was still in the prairies, but on the move. It was flatter than
Alberta, but still, it felt familiar. He came across a point of
interest. It - like most of the drive - was flat, too. Possibly the
point was that the flatness was of particular interest. He didn't know
-- there was no further information.

He carried on, zig-zagging his way northeast, towards his destination,
leaving his former life behind.

Occasionally the roads were good. However, they seemed to have grown
worse after crossing in to the new province. Some went so far as to
have no shoulder. They made him uncomfortable.

Unlike long drives in his youth, this one was draining. Both
physically and emotionally. He missed his friends and the benefits
that came with them. But not enough to go back.

He carried on, in no hurry, through the wind and a slight worry.

There were many small towns along the way. One was a paradise away
from home, or at least that's what they said it was. Who knows what
they meant by that, it seemed barren and deserted. If quiet was
paradise, then it would have been the Caribbean. Minus the people, the
beaches, and the booze.

Soon he would arrive at his destination and begin the task of making a
home. Though he was not sure what that would look like, the idea
appealed to him. This time he'd do it right. He'd get to know people
and step outside his zone of comfort. This excited him and made him
nervous at the same time.

He carried on, reflecting on his new home away from home.

He carried on, motoring through the discomfort of things left behind
and things to come.

He carried on, through the wind swept prairies that were his home.

He carried on.

\newpage

\section{Drive}

He found himself driving across the prairies -- the road stretching as
far as the eye could see. The road undulated: small rises and dips
smoothly rocked the him and the car.

He had drifted off, for who knows how long -- certainly not him. He
was driving south. In the far distance, after where the road
disappeared, the large mountains rose. Like waves in a far off storm,
they towered and terrorized the surroundings.

The mountains looked like the buildings of his downtown, but with
softer edges. Snow gathered there more readily than could be expected
of a high-rise, with its infinitely steep slopes.

He was glad to be away from the city. The noise of life there had
always buzzed busily in his head. Like an insect stuck in a vent. He
had often wished it would die down, that he might get used to it; but,
instead, it continued unabated: pestering his every moment awake and
-- when he was unlucky -- his sleep. He was glad to be away, but the
foreign sight of the mountains in the distance made him uneasy.

The stature of the mountains seemed at odds with the flat demeanour of
the ground beneath him. The sudden ascension of the mountains evoked a
sense of vertigo equalled by his downward high-rise view. The sight
made him uncomfortable, yet still, he drove on.

A small town crept slowly in to view. It was as foreign as the ranges
that dwarfed it further. It had been years since he last drove
through, but the place had not changed. Except the price of gas, which
had soared over the last few decades.

The lull of the road had made him tired. He pulled over at the nearest
gas station and purchased a coffee. After returning to the car he
continued on, unsure of what he would do next, except to continue to
drive towards the rocky pillars that lay ahead, beyond the end of the
road.

\newpage

\section{My Past Self}

It's been a few months since I last used. My past has since come
rushing back. \textit{Rushing}. \textbf{\textit{Rushing back}}.

I lost my youth, my wife and myself. It's at once comically cliche and
viscerally real. I dread the days I relive the past: unable to -- or
not wanting to -- experience misery. I can not help it either way. It
is as if I were there again in person, doing what I will regret in
spite of the consequences. It's \textit{fucking} torture. Torture of a
form I would not inflict on another.

I often wonder if other people do this too. They must... right? I wish
it weren't so easy. Though the fact that it is easy may just be a sign
of dedicated practice. Not a nice thought. I wonder how others relive
their past. I mean, mine isn't always negative, but the more lucid
memories are those of things I did not particularly like. It is
unfortunate I don't dwell as long on the good memories. Though the
only way I could do that is by being ok with thinking how great I
am. Which is not an ideal situation either.

I wish I were able to write candidly about my past, but the editor in
me stops me. I need a blind on the screen. Something to stop me from
being me. Something to interview but not intervene.

% There is always time enough for that later -- or all writers hope
% that there is.

I would write about how I feel this illness has robbed me of
everything, including my future. I no longer care. I no longer care
about my job. I no long care about my house. I no longer care about
anything. Except I care about telling someone all about how I don't
care. It's \textit{fucking} ridiculous. It's \textbf{\textit{absurd}}.

Maybe I need to tell someone so that my mistakes are known. With the
hope that they might be taken as a cautionary tale. Not a tale of loss
or mishap, but of indifference. My own indifference. Towards
myself. It's my only well honed skill: the blade with which I've cut
at away at this happy life, a blade that is as dull as my remaining
senses.

Do not for a minute imagine I believe this story to be of any
significant help. I have wholesale doubt that this that I write is of
any consequence to anyone.  Nor will it steer them in any different
direction than the one they've already picked up momentum on. Yet I
still feel the need to share it, despite its lack of depth.

% I hope these word will prove to be more colourful than straight
% forward.

I fell strung on a noose. A noose as frayed as my wits. As I hang here
suffocating and waiting for a humiliating fall, I wonder about where
the illness begins and where I end.

The last time I used... I saw it coming and I didn't care. I don't
regret the use, though I expected more from it. It is not as
satisfying as it used to be. It is no wonder the saying is cliche:
chasing the dragon and all that. What a load of nonsense: I am not
looking for my first high, I am looking for my next one. I wish that
weren't so. I wish every day life would bring me the satisfaction I
seek. Maybe it does. Maybe this is it.  Maybe there is no replacement
for using. Maybe it is its own goal. Maybe I am just posturing,
finding a way to make it all ok. And maybe using is ok, it's just me
that never will be.

\newpage

\section{Harbouring Hate}

Cautiously and carefully he crept. It was critical he not be seen. His
mark, a car -- a carefully crafted and curated masterpiece of motored
magic -- sat waiting, longing for his touch. This is the part he
loved: the time before the grand exit, when anything seemed
possible.

Further firing his excitement was a crowded celebration above. The
chatter of the crowd reached down the hills, through the trees, over
the grass, and meet his ears with titillation. The chatter was loud
and proud, but meaningless. Large groups congregated and
exaggerated. It was like watching group masterbation, but without any
nervous hesitation. They engaged their own vanity and let loose
stinging statement shaped to shear and decisively destroy other
egos. These are the people for whom he harboured hate.

The lock came easily -- the click of the bolt was orgasmic. Again he
crept, this time around the door and into the seat. It welcomed him
into its embrace, pulling him gracefully into place. He savoured the
moment. Perhaps he could keep it, he wondered... briefly. But the
reality of his humble and hopeless life hummed in is his head. This
was not to be.

He pushed a button and the engine came to life. It purred passionately
as the exhaust poured sensually over everything in its reach. He
reveled in the moment, paralyzed with pleasure. Then he snapped to,
depressed the clutch, shifted down, pressed the gas, and drove off
swiftly into the night. With no hesitation, quickly towards his final
destination.

\newpage

\section{Different People}

I wish I could be different people.

Some would be she, some would be he, some even Cree, but all must be
me. How I wish that could be. Not even just three. Nor must all of
them agree, but all must be me. I haven't the heart to let anyone see,
because as far as I know, there is no one like me.

If this sounds like you, then come be part of my crew. We can start
with just two. Physically, emotionally and presently too. For the good
times and the bad times through.

\newpage

\section{Living Alone}

The world started without him. It continued well after, as well.

He was born in a small rural Alberta town, just south of nowhere and
north of nothing important. His life was full of disappointment,
heart-break, and loved ones lost. Yet still, he enjoyed the days.

In particular, he enjoyed the breeze the would swoop down from the
mountains in the West, curl around his house like a cat circling it's
rest. The wind would crep in to every nook and cranny in the place,
laying silt on every available surface.

He enjoyed the creak of the old wood and the whistle that came from
the tin roof that had long since become loose from the constant gusts
across the prairies.

On most nights, the sound soothed him off to sleep. But not this evening.

He had been awake since 6 am, just before the sun had broken. But even
in the lateness of evening, he couldn't find a nod. His mind was
fixating upon the things that had gone wrong in his life.

He had not been a social man. He had picked the house and the land he
inhabited to escape the city he had once visited for 10 years in his
youth. But the move had not always worked the trick he had hoped. His
mind still returned there, not letting him escape the tendrils that
had wound around his past, urban life.

He thought of the man for whom the department store's doors had caught
his coat as he, the stranger, had been exiting with gifts piled high
for his children. He recalled the urge to help, the sense that it
would be polite, and even kind, to reach out and pull the door open so
as to release the man's coat from it's clutches. He also recalled how
he had simply moved on, not fulfilling his imagined nicety.

This moment had become a fixture in his frequent visits to the
past. He had no good reason to think this made him a bad person, one
lacking in a certain type of character. Yet still, it came back, over
and over again -- revisiting his guilt upon him.

Things like this occupied his mind a lot, and usually for no
particular reason, either.

He had had a good day that day. He had cleaned the kitchen for the 3rd
time that month, and scrubbed the bathroom to a shimmer for the
8th. He liked the order, but lacked the discipline to leave good
enough alone.

This was they way for him, the visits to the past came at least once a
week, but this evening was -- in particular -- unsatisfactorily
unpleasant.

He rose out of bed and made his way to the kitchen for a glass of
milk, hoping that it might help calm his mind. Over the course of the
short walk from the bed to the fringe, he quietly committed himself to
returning to the city.

He hoped that his time in the house, in the wind, with it's creeks,
and it's groans, dropped in the prairies, just east of the mountains,
would not return to him during a sleepless night in the city.

\newpage

\section{Fast Food}

He awoke with a fright. It was still night, but the moon was beaming,
so it was light. He decided he needed a bite, so he got up and went to
McDonald's -- you know, to get it right.  After picking up his food,
he pulled over to the right, avoiding the traffic that made him want
to fight. You know the type, right? It clouds the mind and blurs the
sight. He bit down on the burger with his jaw’s might. The ketchup
erupted through the bun, it screamed through the air as it fell to his
lap, leaving him covered; the red against his denim engendered a sense
of spite. Fuck this night, with it’s moon so bright. I still want to
fight! Especially at this sight!

Some say rhyme might be trite, but I don’t think they are right. Not
at least for tonight.

\newpage

\section{Better Times}

%Purple noises erupt.
%Cavernous caves corrupt.

The land shifts, shuffling the surface and erasing structures. Time
stops. Life is paused, but not forgotten. Matter melts and
consciousness turns over. Feelings are lost, but fear -- where it can
be seen -- runs rampant. The light dims and the dark looms, longingly
holding onto space. Stars gaze at the destruction, illuminating the
remains. Bodies rot and feed reminders of better times. Heaven holds
its share, but overflows; hell harbours the rest. We did it to
ourselves, we must remember that.

There was a time before this time that was at peace. We remember this,
but forget the why's and the how's of our present. The world lives
on. Unrelenting.

It continues to feed us, this earth that remains. But not well.

We continue to live. But not well.

Yet there is hope: glimmers in the eyes of our children. Like stars
looking up at possibilities.

We continue to live. Not with our own hope. But with that of our
future selves.

We continue to live.

\newpage

\section{Vices in Volume}

%Passing perfectly puraied poultry. Perhaps this is the pure purpose
%posed by prophets.

Creep carefully and cautiously, craft and create costumes then master
making masks to hide your person. Your person is valuable.

%Fast, and furiously find foliage for fires that burn white.

%There must be more, or less, to manage masterfully.

Music makes mountains more like vallies. Vallies make mountains more
like monuments: magnanimous, magistic and mystical.

Vultures swoop down vacuously vying for voluptuous vixens.

Vices, in volume, hurt. Praise, when possible, improves posture,
preserves and protects.

Magic make moments more meaningful than science. But science succeeds
in curing cancerous cravings, cavities and the flesh corrupted by
blades.

Time timbers and toppels all: just wait while wonder wipes worry
away. Erase the tears and extinguish fears.

People pleasers pursue pleasure purposely but imperfectly. Your peers
help provide purpose, pleasure and perspective.

\newpage

\section{Softly Sleeping}

The hiss of the hose hummed in his head. ``Soon you will dead,'' it
said.

He sat in the bath tub with a bag on his head and a hose at the
ready. His hand on the nozzle calmly cured the angst of his life,
leaving nothing but regret.

It had been a miserable life for him, as far as misery goes. He'd
never had a good head on his shoulders, never got what others saw. He
was confined to the simple life, and told he should be content with
what he had. But he had nothing -- and he knew it.

\textit{Nothing worth anything}, he thought.

\textit{Actually, I do have something}, he corrected
himself. \textit{I have plenty of regret, that's what I have. And
  anger -- that too}, he mused.

He was right, in a way. Nothing had come easy to him; but somethings
had come, nevertheless. These thoughts were useless now though, not
enough reverse his pending actions. No miracle would save his soul, no
soul could save his life.

\textit{It's done, it's decided, nothing can stop it now}, echoed a
voice much like his own.

The hiss of the hose hummed in his head. ``You're dead,'' it
said. Slowly from then, he slipped silently to sleep.

\newpage

\section{Fridge}

Fridge awoke as he usually did. Cold on the inside and warm on the
outside. Not cold in the evil way: he was just doing his job. He liked
this job very much, and was very good at it too.

For a few years now, he had hummed in approval of his owner's
taste. Filled with vegetables, meats and -- on occasion -- a tub of
ice-cream, he had circulated cold air reliably to keep his contents
cold. Today was no exception.

Boy awoke unusually: he was warm in the inside but cold on the
outside. Winter had arrived overnight and it there was a chill in the
air. Boy got dressed: he put on a sweater and hat. Then Boy bolted
down the stairs to the kitchen, where Fridge was humming idly. Boy
wanted a mug of hot chocolate. Hot chocolate always warned Boy up:
from the inside out. Boy liked the taste, too, of course.

Dads awoke. They awoke as they had begun to wake in their old age: a
little tired, sore and stiff. Dads heard Boy's heavy footsteps and
almost yelled out for him to stop galloping about the house. But Dads
too felt the chill and let Boy run.

Eventually Dads joined Boy and Fridge in the kitchen, and sat down
around the breakfast table. One Dad had fried back-bacon and eggs,
while the other had grilled some toasts. Boy had made a round of hot
chocolate. He, himself, was on his second.

They sat there eating breakfast lazily. It was a Saturday morning, and
the trees outside were swaying in the wind. There was no hurry to get
going this morning, and the weather outside was no motivator.


\end{document}
